\documentclass{screenplay}
\usepackage[utf8]{inputenc}

\title{O Destino de Miguel}
\author{A Internet}

\begin{document}
\coverpage
\fadein

Abertura ao som de YMCA.

\begin{dialogue}{Narrador (V.O.)}
    Pelotas, 1724. Numa época em que só se cagava para fora, as bolas não se encontravam e a cidade não fazia jus ao nome. Enfim, numa época em que só tinha macho, tchê.
\end{dialogue}

O condado medieval em que se passa a nossa história passa a ser mostrado. Nada demais acontece até então. Apenas pessoas comuns cuidando dos seus afazeres cotidianos.

\begin{dialogue}{Narrador (V.O.)}
    Nasce um avantajado bebê.
\end{dialogue}

\begin{dialogue}[com a voz mais suave a cada sentença]{Narrador (V.O.)}
    Quando criança foi visitado por um anjo. Um anjo lindo, com olhos azuis, asas rosas, livre para voar.
\end{dialogue}

O NARRADOR tosse ao perceber a mudança na voz e corrige as últimas palavras.

\begin{dialogue}[já com a voz normal]{Narrador (V.O.)}
    Para voar.
    Depois dessa misteriosa aparição nada poderá deter O DESTINO DE MIGUEL
\end{dialogue}

Continua a abertura ao som de YMCA

\intslug[dia]{Casa de Miguel}

MIGUEL está calmamente em sua cama calçando seus sapatos.

JUPARÁ entra na casa de MIGUEL apressado

\begin{dialogue}{Jupará}
    Miguel! Miguel! Abra esta porta.
\end{dialogue}

JUPARÁ olha para cima e encontra MIGUEL.

MIGUEL continua na cama calçando calmamente os sapatos

\begin{dialogue}[continua ao avistar Miguel]{Jupará}
    Aí está você seu desgraçado, eu preciso falar com você imediatamente. A cidade está em polvorosa, Miguel\ldots 
    Onde está aquele maldito telefone? O telefone do psicólogo, você sabe. \paren{aumenta o tom de voz} Você sabe do que eu estou falando, do psicólogo!
\end{dialogue}

\begin{dialogue}{Miguel}
    Não preciso disso
\end{dialogue}

\begin{dialogue}{Jupará}
    Escute, escute
\end{dialogue}

\begin{dialogue}{Miguel}
    Não preciso. Eu sou bom da cabeça, Jupará.
\end{dialogue}

JUPARÁ senta-se cansado

\begin{dialogue}{Miguel}
    E as pessoas falam demais
\end{dialogue}

\begin{dialogue}{Jupará}
    Hein?
\end{dialogue}

\begin{dialogue}{Miguel}
    Eu sou uma pessoa livre
\end{dialogue}

JUPARÁ demonstra não acreditar no que MIGUEL diz.

MIGUEL pula de sua cama

\begin{dialogue}[em tom de deboche]{Miguel}
    Eu faço o que eu quiser, meu amigo. E não vou parar.
\end{dialogue}

\begin{dialogue}{Jupará}
    O que você quer dizer com "não vou"?
\end{dialogue}

\begin{dialogue}{Miguel}
    Ah, Jupará, você sabe. Você me conhece!
    Eu não vou parar de comer cu. Ouviu?
\end{dialogue}

MIGUEL chuta o banco no qual JUPARÁ apoiava os pés, e esse exprime um grito de dor.

\begin{dialogue}{Jupará}
    Não chute a maldita cadeira. Escute o que eu tenho para lhe falar, escute.
\end{dialogue}

MIGUEL sai de sua casa e JUPARÁ o segue tentando convencê-lo.

\extslug[dia]{Vila de Pelotas. Ouve-se ao fundo o barulho das pessoas conversando nas ruas}

MIGUEL segue na frente apressado e JUPARÁ tenta acompanhar seus passos

\begin{dialogue}{Jupará}
    A polícia está atrás de você, e os cus que você andou comendo na última semana, todos foram reclamar ao prefeito.
    Você tem que sair daqui imediatamente, ou então parar com isso. As pessoas estão nervosas.
    Esta história de você ficar comendo o cu dos outros\dots~
    Olha, o meu tudo bem, eu sou seu amigo
\end{dialogue}

MIGUEL, visivelmente irritado com as declarações de JUPARÁ, continua sua caminhada se lhe dar muita atenção.

\begin{dialogue}[já irritado com a indiferença de MIGUEL]{Jupará}
    Olha pra mim quando eu falar com você! Olha pra mim!
\end{dialogue}

MIGUEL se volta para JUPARÁ

\begin{dialogue}{Miguel}
    Você está se comportando como uma tia velha.
\end{dialogue}

\begin{dialogue}[um pouco mais calmo]{Jupará}
    Você sabe que a polícia tem cães farejadores, né? E os cães não poupam ninguém. Você sabe bem disso. Agora\dots~
    Por favor, escute o que estou falando. Eu preciso lhe contar. Os cães eles\dots~
\end{dialogue}

\begin{dialogue}[interrompe JUPARÁ]{Miguel}
    Eu só vejo cu, só penso em cu, Jupará.
    Obrigado pela sua atenção, mas\dots~ você sabe\dots
\end{dialogue}

\begin{dialogue}{Jupará}
    Obrigado, obrigado, mas não é isso que eu estou falando do psicólogo.
    Sim, o psicólogo Brian.
\end{dialogue}

\begin{dialogue}[impaciente]{Miguel}
    Aahh
\end{dialogue}

\begin{dialogue}{Jupará}
    Ele é o melhor psicólogo.
    Que porra de pilastra.
    Escute o que o que estou falando. Por favor, vamos ao psicólogo comigo. Ele é meu amigo.
\end{dialogue}

\begin{dialogue}[com um olhar desafiador]{Miguel}
    Pra quê?
\end{dialogue}

MIGUEL e JUPARÁ param de andar.

Neste momento o PADRE local começa a discursar em voz alta para a população enquanto MIGUEL e JUPARÁ continuam conversando. O discurso do padre pode ser ouvido ao fundo, muitas vezes ao mesmo tempo em que MIGUEL e JUPARÁ conversam.

\begin{dialogue}{Padre}
    Atenção!
\end{dialogue}

\begin{dialogue}[continuando]{Miguel}
    Eu já comi, Jupará.
\end{dialogue}

\begin{dialogue}{Padre}
    Atenção população do condado.
\end{dialogue}

\begin{dialogue}[impressionado com o que Miguel acabara de dizer]{Jupará}
    Não. O psicólogo? Não, não acredito.
\end{dialogue}

\begin{dialogue}{Miguel}
    Comi, comi.
\end{dialogue}

\begin{dialogue}{Padre}
    Eu tenho um comunicado muito importante pra fazer pra todos vocês.
\end{dialogue}

\begin{dialogue}{Miguel}
    E do jeito que a minha lista está, meu amigo, ele já é coisa do passado.
\end{dialogue}

\begin{dialogue}[se vira transtornado]{Jupará}
    Não, ah não!
\end{dialogue}

\begin{dialogue}{Padre}
    Um sujeito \paren{indistinguível} andando por aí de noite.
\end{dialogue}

\begin{dialogue}[se volta para Miguel]{Jupará}
    A cidade quer a sua cabeça e você está brincando.
\end{dialogue}

\begin{dialogue}{Padre}
    Um rapazinho que dever ter seus metro e oitenta de altura\dots
\end{dialogue}

\begin{dialogue}{Miguel}
    Você não viu nada.
\end{dialogue}

MIGUEL vira-se de costas para JUPARÁ e torna a andar. JUPARÁ sai no seu encalço.

\begin{dialogue}{Padre}
    \dots uma lapa de testa, e com um pirocão desse tamanho.
\end{dialogue}

MIGUEL e JUPARÁ, até então alheios ao que acontecia, observam o PADRE, que continua o seu discurso.

\begin{dialogue}{Padre}
    É uma anomalia! E sabe o que ele quer minha gente?
\end{dialogue}

MIGUEL e JUPARÁ \pov

O PADRE estica o braço esquerdo como se estivesse apontando para um lugar inexistente.

\begin{dialogue}{Padre}
    Ele quer comer CU!
    Primeiro ele enfia o dedo, assim ó. Ó, ó, mas enfia.
\end{dialogue}

O PADRE abaixa o braço

\begin{dialogue}{Padre}
    E depois diga adeus suas pregas.
\end{dialogue}

Câmera volta para MIGUEL

\begin{dialogue}{Padre}
    E depois diga adeus suas pregas.
\end{dialogue}

MIGUEL continua sua caminhada pensativo, tentando lembrar de algo.

\begin{dialogue}{Padre}
    E não adianta gritar não, que ele mete.
\end{dialogue}

\begin{dialogue}[ao se lembrar, diz baixinho]{Miguel}
    Eh, eu já comi esse pastor
\end{dialogue}

As palavras do PADRE voltam a ser plano de fundo da conversa entre MIGUEL e JUPARÁ.

\begin{dialogue}{Padre}
    E não sai de casa, pelo amor de deus\dots
\end{dialogue}

\begin{dialogue}[chamando Miguel, que está a frente]{Jupará}
    Ei, ei! Como assim?
\end{dialogue}

\begin{dialogue}{Padre}
    \dots fique com a bunda na parede.
\end{dialogue}

\begin{dialogue}[entrando por uma porta]{Miguel}
    Ajoelhou tem que rezar, meu amigo.
\end{dialogue}

\begin{dialogue}{Padre}
    Porque o bixo vai pegar.
\end{dialogue}

JUPARÁ para à porta e observa algumas placas

\begin{dialogue}{Jupará}
    Que merda é essa?
\end{dialogue}

\intslug[dia]{Na sala do psicólogo}

\fadeout
\theend
\end{document}
